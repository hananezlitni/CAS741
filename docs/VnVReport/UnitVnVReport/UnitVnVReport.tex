\documentclass[12pt, titlepage]{article}

\usepackage{xr}
\usepackage{color,soul}
\usepackage{booktabs}
\usepackage{tabularx}
\usepackage{hyperref}
\usepackage[tablename=Table,aboveskip=5pt,belowskip=5pt]{caption}
\hypersetup{
    colorlinks,
    citecolor=blue,
    filecolor=magenta,
    linkcolor=black,
    urlcolor=cyan
}
\usepackage[round]{natbib}

%% Comments

\usepackage{color}

\newif\ifcomments\commentstrue

\ifcomments
\newcommand{\authornote}[3]{\textcolor{#1}{[#3 ---#2]}}
\newcommand{\todo}[1]{\textcolor{red}{[TODO: #1]}}
\else
\newcommand{\authornote}[3]{}
\newcommand{\todo}[1]{}
\fi

\newcommand{\wss}[1]{\authornote{blue}{SS}{#1}}
\newcommand{\hz}[1]{\authornote{magenta}{Author}{#1}}


\externaldocument{../../SRS/CA}
\externaldocument{../../VnVPlan/SystVnVPlan/SystVnVPlan}
\externaldocument{../../VnVPlan/UnitVnVPlan/UnitVnVPlan}
\externaldocument{../../Design/MIS/MIS}
\externaldocument{../SystVnVReport/SystVnVReport}

\newcommand{\progname}{Library of Simplex Method Solvers}
\newcommand{\famname}{LoSMS}

\begin{document}

\title{System Verification and Validation Test Report: A \progname{} 
(\famname{})} 
\author{Hanane Zlitni}
\date{December 18, 2018}
	
\maketitle

\pagenumbering{roman}

\section{Revision History}

\begin{tabularx}{\textwidth}{p{3cm}p{2cm}X}
\toprule {\bf Date} & {\bf Version} & {\bf Notes}\\
\midrule
December 15, 2018 & 1.0 & First draft\\
\bottomrule
\end{tabularx}

~\newpage

\section{Symbols, Abbreviations and Acronyms}

See System V\&V Report found at: 
\url{https://github.com/hananezlitni/HZ-CAS741-Project/tree/master/docs/VnVReport/SystVnVReport/SystVnVReport.pdf}.

\newpage

\tableofcontents

\listoftables %if appropriate

\newpage

\pagenumbering{arabic}

This document reports all results obtained from testing the \progname{} 
(\famname{}) tool. The test cases under evaluation, along with the full 
documentation of the library, can be found at: 
\url{https://github.com/hananezlitni/HZ-CAS741-Project} \\

The document starts by outlining the results of the test cases related to the 
tool's functional requirements in Section \ref{FunReqsEval}. Then, the results 
of the test cases related to the non-functional requirements are reported in 
Section \ref{NonFunReqsEval}. This is followed by Sections 
\ref{Comparison}, \ref{unitTesting} and \ref{ChangesAfterTesting}, which 
discuss comparing any previous implementations with the existing one, unit 
testing of the library and changes which occured after testing, respectively. 
Section \ref{AutomatedTest} discusses the tests that were executed 
automatically, while sections \ref{TraceReq} and \ref{TraceMod} provide 
traceability tables for the tests with the requirements and modules. Sections 
\ref{codeCoverage} concludes the document by discussing the metrics used to 
achieve code coverage.

\section{Functional Requirements Evaluation} \label{FunReqsEval}

See Section \ref{unitTesting}.

\section{Nonfunctional Requirements Evaluation} \label{NonFunReqsEval}

This section will report the accuracy of the outputs obtained from executing 
the test cases in the Unit V\&V Plan. \\

The evaluation of the rest of the qualities were detailed in the System V\&V 
Report found at: 
\url{https://github.com/hananezlitni/HZ-CAS741-Project/tree/master/docs/VnVReport/SystemVnVReport/SystemVnVReport.pdf}.

\subsection{Accuracy}

T\ref{Accuracy} in the System V\&V Plan tests the accuracy of the solutions 
produced by \famname{}. \\

The following table reports the relative errors of the expected outputs for 
each test case detailed in the Unit V\&V Plan that produces numbers. \\

\noindent
\begin{tabularx}{\textwidth}{p{1.5cm}p{3cm}p{4.5cm}p{6cm}X}
	\toprule {\bf T} & {\bf Expected} & {\bf Actual} & {\bf Relative Error}\\
	\midrule
	T\ref{solveLpMax1} & $Z$= 1052000 \newline
	$x_1$= 4 \newline
	$x_2$= 10 \newline
	$x_3$= 14 \newline
	& $Z$= 1052000.0 \newline
	$x_1$= 4.0 \newline
	$x_2$= 10.0 \newline
	$x_3$= 14.0 \newline 
	& $\epsilon_Z$ = 0 \newline
	$\epsilon_{x_1}$ = 0 \newline 
	$\epsilon_{x_2}$ = 0 \newline 
	$\epsilon_{x_3}$ = 0 \newline  
	\\
	T\ref{solveLpMax2} & $Z$= 180 \newline
	$x_1$= 40 \newline
	$x_2$= 30 \newline
	&$Z$= 180.0 \newline
	$x_1$= 40.0 \newline
	$x_2$= 30.0 \newline
	& $\epsilon_Z$ = 0 \newline
	$\epsilon_{x_1}$ = 0 \newline 
	$\epsilon_{x_2}$ =  0\newline
	\\
	\bottomrule
\end{tabularx}
\captionof{table}{Relative Errors of Expected Outputs}
	
\section{Comparison to Existing Implementation}	 \label{Comparison}

This section is not applicable for \famname{}.

\section{Unit Testing} \label{unitTesting}

T\ref{solveLpMax1} \& T\ref{solveLpMax2} in the Unit V\&V Plan verify the 
\textit{solveLP()} function for 2 maximization linear programs. The addition of 
these 2 test cases was intended to have a thorough coverage of the cases. Both 
test have passed.\\

T\ref{solveLpMin} in the Unit V\&V Plan verifies the \textit{solveLP()} 
function for a minimization linear program that doesn't have an optimal 
solution. The test has passed.\\

The Unit V\&V Plan can be found at: 
\url{https://github.com/hananezlitni/HZ-CAS741-Project/tree/master/docs/VnVPlan/UnitVnVPlan/UnitVnVPlan.pdf}.

\section{Changes Due to Testing} \label{ChangesAfterTesting}

No changes besides what was detailed in the System V\&V Report were made.

\section{Automated Testing} \label{AutomatedTest}

All for functional requirements were automated using \textit{unittest} provided 
by Python. The implemented tests are executed by entering a command in the 
command line tool.
		
\section{Trace to Requirements} \label{TraceReq}

Table \ref{Table:ReqTraceability} shows the traceability between the test cases 
in the System V\&V Plan and the requirements.

\begin{table} [h!]
	\centering
	\begin{tabular}{|c|c|}
		\hline	
		\textbf{T} & \textbf{Requirements}\\
		\hline 
		T\ref{solveLpMax1}& R\ref{R_Inputs}, R\ref{R_CanonicalForm}, 
		R\ref{R_Calculate}, R\ref{R_Output}\\ \hline
		T\ref{solveLpMax2}& R\ref{R_Inputs}, R\ref{R_CanonicalForm}, 
		R\ref{R_Calculate}, R\ref{R_Output}\\ \hline
		T\ref{solveLpMin}& R\ref{R_Inputs}, R\ref{R_CanonicalForm}, 
		R\ref{R_Calculate}, R\ref{R_Output}\\ \hline
	\end{tabular}
	\caption{Traceability Between Test Cases and Requirements}
	\label{Table:ReqTraceability} 
\end{table}
		
\section{Trace to Modules} \label{TraceMod}	

Table \ref{Table:ModTraceability} shows the traceability between the test cases 
and the modules. \\

\begin{table} [h!]
	\centering
	\begin{tabular}{|c|c|}
		\hline	
		\textbf{T} & \textbf{Modules}\\
		\hline 
		T\ref{solveLpMax1}& Simplex Method Solver Module \\ \hline
		T\ref{solveLpMax2}& Simplex Method Solver Module \\ \hline
		T\ref{solveLpMin}& Simplex Method Solver Module, Exceptions Module \\ 
		\hline
	\end{tabular}
	\caption{Traceability Between Test Cases and Modules}
	\label{Table:ModTraceability} 
\end{table}

\newpage 

\section{Code Coverage Metrics} \label{codeCoverage}

It was important to ensure that the implemented test cases were achieving the 
highest possible code coverage. To verify this, Python's \textit{Coverage.py} 
tool was used. \\

The tested linear program was: \\

min$\;Z\;=\;-2x_1\;+\;3x_2$\newline
$s.\;t.$ $\hspace*{0.5cm} 3x_1\;+\;4x_2\; = \;24$\newline
$\hspace*{1.2cm} 7x_1\;+\;4x_2\; = \;16$\\

Table 4 reports the code coverage that the test cases achieved:\\

\noindent
\begin{tabularx}{\textwidth}{p{5cm}p{3cm}p{2.5cm}p{3cm}X}
	\toprule {\bf Module} & {Statements} & {Missing} & {Coverage} \\
	\midrule
	Exceptions.py & 6 & 0 & 100\% \\
	
	SimplexSolverADT.py & 131 & 1 & 99\% \\
	
	SimplexSolverADT\_Test.py & 14 & 0 & 100\% \\
	
	Total & 151 & 1 & 99\% \\
	
	\bottomrule
\end{tabularx}
\captionof{table}{Code Coverage Results by Coverage.py}

~\newpage

\bibliographystyle{plainnat}
\bibliography{../../../refs/References}

\end{document}