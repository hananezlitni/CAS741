\documentclass[12pt, titlepage]{article}

\usepackage{xr}
\usepackage{color,soul}
\usepackage{booktabs}
\usepackage{tabularx}
\usepackage{hyperref}
\usepackage[tablename=Table,aboveskip=5pt,belowskip=5pt]{caption}
\hypersetup{
    colorlinks,
    citecolor=blue,
    filecolor=magenta,
    linkcolor=black,
    urlcolor=cyan
}
\usepackage[round]{natbib}

%% Comments

\usepackage{color}

\newif\ifcomments\commentstrue

\ifcomments
\newcommand{\authornote}[3]{\textcolor{#1}{[#3 ---#2]}}
\newcommand{\todo}[1]{\textcolor{red}{[TODO: #1]}}
\else
\newcommand{\authornote}[3]{}
\newcommand{\todo}[1]{}
\fi

\newcommand{\wss}[1]{\authornote{blue}{SS}{#1}}
\newcommand{\hz}[1]{\authornote{magenta}{Author}{#1}}


\externaldocument{../../SRS/CA}
\externaldocument{../../VnVPlan/SystVnVPlan/SystVnVPlan}
\externaldocument{../../Design/MIS/MIS}

\newcommand{\progname}{Library of Simplex Method Solvers}
\newcommand{\famname}{LoSMS}

\begin{document}

\title{System Verification and Validation Test Report: A \progname{} 
(\famname{})} 
\author{Hanane Zlitni}
\date{December 18, 2018}
	
\maketitle

\pagenumbering{roman}

\section{Revision History}

\begin{tabularx}{\textwidth}{p{3cm}p{2cm}X}
\toprule {\bf Date} & {\bf Version} & {\bf Notes}\\
\midrule
December 15, 2018 & 1.0 & First draft\\
\bottomrule
\end{tabularx}

~\newpage

\section{Symbols, Abbreviations and Acronyms}

The following symbols, abbreviations and acronyms were used in this document: \\

\renewcommand{\arraystretch}{1.2}
\begin{tabular}{l l} 
  \toprule		
  \textbf{symbol} & \textbf{description}\\
  \midrule 
  \famname{} & Library of Simplex Method Solvers\\
  T & Test\\
  R & Requirements\\
  NFR & Nonfunctional Requirements\\
  V\&V & Verification and Validation\\
  $s.\;t.$ & Subject to\\
  $Z$ & Optimal solution of the objective function\\
  $x_1, x_2, x_3$ & Decision variables\\
  $\epsilon_{subscript}$ & The relative error of the subscript\\
  s & seconds\\
  \bottomrule
\end{tabular}\\

\newpage

\tableofcontents

\listoftables %if appropriate

\newpage

\pagenumbering{arabic}

This document reports all results obtained from testing the \progname{} 
(\famname{}) tool. The test cases under evaluation, along with the full 
documentation of the library, can be found at: 
\url{https://github.com/hananezlitni/HZ-CAS741-Project} \\

The document starts by outlining the results of the test cases related to the 
tool's functional requirements in Section \ref{FuncReqsEval}. Then, the results 
of the test cases related to the non-functional requirements are reported in 
Section \ref{NonFuncReqsEval}. This is followed by Sections 
\ref{ComparisonImplem}, \ref{UnitTesting} and \ref{ChangesDueToTesting}, which 
discuss comparing any previous implementations with the existing one, unit 
testing of the library and changes which occured after testing, respectively. 
Section \ref{AutomatedTesting} discusses the tests that were executed 
automatically, while sections \ref{ReqTrace} and \ref{ModTrace} provide 
traceability tables for the tests with the requirements and modules. Sections 
\ref{CodeCoverage} concludes the document by discussing the metrics used to 
achieve code coverage.

\section{Functional Requirements Evaluation} \label{FuncReqsEval}

T\ref{Max_Opt} \& T\ref{Max_NoOpt} in the System V\&V Plan verified the cases 
where a maximization linear program has and doesn't have an optimal solution. 
Both test cases have passed. \\

T\ref{Min_Opt} \& T\ref{Min_NoOpt} in the System V\&V Plan verified the cases 
where a minimization linear program has and doesn't have an optimal solution. 
Both test cases have passed. \\

T\ref{NoObjcFunc} \& T\ref{NoLCs} in the System V\&V Plan verified the cases 
where the objective function and linear constraints, respectively, are missing. 
Both test cases have passed. \\

The System V\&V Plan can be found at: 
\url{https://github.com/hananezlitni/HZ-CAS741-Project/blob/master/docs/VnVPlan/SystVnVPlan/SystVnVPlan.pdf}.

\section{Nonfunctional Requirements Evaluation} \label{NonFuncReqsEval}

\subsection{Usability}

A usability study was conducted to check whether \famname{} user-friendly or 
not (T\ref{Usability} in the System V\&V Plan). \\

The participant, Soumaya Zlitni, was given the linear program to solve after 
reading the information provided in the project repository's README.md file 
(found at: 
\hz{https://github.com/hananezlitni/HZ-CAS741-Project/blob/master/src/README.md}).
 At the end of the usability testing, Soumaya filled in the usability survey 
 that was handed to her. The filled survey can be found in the appendix 
 (Section \ref{UsabilityTestingQs}). \\
 
 There might have been some bias in the usability testing since the participant 
 is a relative.
		
\subsection{Portability}

T\ref{Portability} in the System V\&V Plan specified the 3 operating systems 
that \famname{} should be operable on: Windows, MacOS and Linux. \\

The tool was ran twice in each operating system: once with the linear program 
in T\ref{Max_Opt} in the System V\&V Plan and once with the same linear program 
but with a missing input (the linear program goal). This was done to test the 
behaviour of the library in the correct and incorrect conditions in each 
operating system. \\

Running the library on Windows and Linux specified the Python libraries that 
the tool depends on to execute without errors. The details of the dependencies 
and how to download them are explained in the README.md file in the project's 
GitHub repository, found at: 
\url{https://github.com/hananezlitni/HZ-CAS741-Project/blob/master/src/README.md}.
 The portability test passed for all 3 
operating systems by printing the correct optimum in the first execution and 
the missing input exception message in the second execution. \\

\subsection{Accuracy}

T\ref{Accuracy} in the System V\&V Plan tests the accuracy of the solutions 
produced by \famname{}. \\

The following table reports the relative errors of the expected outputs for 
each test case detailed in the System V\&V Plan that produces numbers. \\

\noindent
\begin{tabularx}{\textwidth}{p{1.5cm}p{3cm}p{4.5cm}p{6cm}X}
	\toprule {\bf T} & {\bf Expected} & {\bf Actual} & {\bf Relative Error}\\
	\midrule
	T\ref{Max_Opt} & $Z$= 3 \newline
					 $x_1$= 0 \newline
					 $x_2$= 0 \newline
					 $x_3$= 3 \newline
				   & $Z$= 3.00 \newline
				     $x_1$= 0 \newline
				     $x_2$= 0 \newline
				     $x_3$= 3.0 \newline 
				   & $\epsilon_Z$ = 0 \newline
				     $\epsilon_{x_1}$ = 0 \newline 
				     $\epsilon_{x_2}$ = 0 \newline 
				     $\epsilon_{x_3}$ = 0 \newline  
				  \\
	T\ref{Min_Opt} & $Z$= -4.57 \newline
					 $x_1$= 2.29 \newline
					 $x_2$= 0 \newline
				   & $Z$= -4.5714 \newline
					 $x_1$= 2.2857 \newline
					 $x_2$= 0 \newline
				   & $\epsilon_Z$ = 0.0003 \newline
					 $\epsilon_{x_1}$ = 0.00188 \newline 
					 $\epsilon_{x_2}$ =  0\newline
				  \\ 
	\bottomrule
\end{tabularx}
\captionof{table}{Relative Errors of Expected Outputs}

\subsection{Correctness}

T\ref{Correctness} in the Systen V\&V Plan tests the correctness of the 
library. The test was done by solving the same linear program using \famname{} 
and MatLab, then comparing the output of both programs. \\

The following table reports the result: \\

\noindent
\begin{tabularx}{\textwidth}{p{5.5cm}p{4cm}p{4cm}X}
	\toprule {\bf Problem} & {\bf \famname{}} & {\bf MatLab} \\
	\midrule
	min$\;Z\;=\;-x_1\;-\;2x_2$\newline
	$s.\;t.$$\hspace*{0.5cm} 2x_1\;+\;x_2\; = \;10$\newline
	$\hspace*{1.2cm} x_1\;+\;x_2\; = \;6$\newline
	$\hspace*{1.2cm} -x_1\;+\;x_2\; = \;2$\newline
	$\hspace*{1.2cm} -2x_1\;+\;x_2\; = \;1$\newline
	
	& $Z$= -10.00 \newline
	  $x_1$= 2.0 \newline
	  $x_2$= 4.0 \newline
	& $Z$= -10.00 \newline
	  $x_1$= 2.00 \newline
	  $x_2$= 2.00 \newline
	\\
	\bottomrule
\end{tabularx}
\captionof{table}{Output Correctness for \famname{} and MatLab}

\subsection{Performance}

T\ref{Performance} in the System V\&V Plan tests the peformance of the library. 
The test was done by solving the same linear program using \famname{} and 
MatLab, then comparing the elapsed time (\textit{time at the start of the 
execution} - \textit{time at the end of the execution}) for both programs. \\

The following table reports the result: \\

\noindent
\begin{tabularx}{\textwidth}{p{5.5cm}p{4cm}p{4cm}X}
	\toprule {\bf Problem} & {\bf \famname{}} & {\bf MatLab} \\
	\midrule
	min$\;Z\;=\;-x_1\;-\;2x_2$\newline
	$s.\;t.$$\hspace*{0.5cm} 2x_1\;+\;x_2\; = \;10$\newline
	$\hspace*{1.2cm} x_1\;+\;x_2\; = \;6$\newline
	$\hspace*{1.2cm} -x_1\;+\;x_2\; = \;2$\newline
	$\hspace*{1.2cm} -2x_1\;+\;x_2\; = \;1$\newline
	
	& 0.000549 s
	& 0.054326 s
				\\
	\bottomrule
\end{tabularx}
\captionof{table}{Elapsed Time for Executions of \famname{} and MatLab}

\subsection{Stability}

Stress testing using a third-party tool was planned for \famname{} to verify 
its behaviour under heavy load. However, due to not finding a tool that 
automatically loads a very large number of inputs to the library, the other 
tests were prioritized and this one was moved to a second phase of testing 
using other methods.
	
\section{Comparison to Existing Implementation}	 \label{ComparisonImplem}

This section is not applicable for \famname{}.

\section{Unit Testing}  \label{UnitTesting}

The unit testing of the \famname{} tool is discussed in details in the Unit 
V\&V Report found at:
\url{https://github.com/hananezlitni/HZ-CAS741-Project/tree/master/docs/VnVReport/UnitVnVReport/UnitVnVReport.pdf}.

\section{Changes Due to Testing}  \label{ChangesDueToTesting}

Initially, The output of \famname{} was a string consisting of the optimal 
solution and the points where it occurs. The string had to be changed to a 
\textit{dictionary} (see 
\url{https://docs.python.org/3/tutorial/datastructures.html#dictionaries}) 
because the assert function in Unittest compared dictionaries and not strings. 
\\

The test cases for missing inputs in the System V\&V Plan initially failed 
because there was a bug in the condition that checked for missing inputs and 
was producing an incorrect output. Upon modifying the condition, the correct 
output was produced and the tests have passed.

\section{Automated Testing} \label{AutomatedTesting}

All the test cases for functional requirements were automated using Unittest 
provided by Python. The implemented tests are executed by entering a command in 
the command line tool.
		
\section{Trace to Requirements} \label{ReqTrace}

Table \ref{Table:Traceability} shows the traceability between the test cases in 
the System V\&V Plan and the requirements.

\begin{table} [h!]
	\centering
	\begin{tabular}{|c|c|}
		\hline	
		\textbf{T} & \textbf{Requirements}\\
		\hline 
		T\ref{Max_Opt}& R\ref{R_Inputs}, R\ref{R_CanonicalForm}, 
		R\ref{R_Calculate}, R\ref{R_Output}\\ \hline
		T\ref{Max_NoOpt}& R\ref{R_Inputs}, R\ref{R_CanonicalForm}, 
		R\ref{R_Calculate}, R\ref{R_Output}\\ \hline
		T\ref{Min_Opt}& R\ref{R_Inputs}, R\ref{R_CanonicalForm}, 
		R\ref{R_Calculate}, R\ref{R_Output}\\ \hline
		T\ref{Min_NoOpt}& R\ref{R_Inputs}, R\ref{R_CanonicalForm}, 
		R\ref{R_Calculate}, R\ref{R_Output}\\ \hline
		T\ref{NoObjcFunc}& R\ref{R_Inputs}, R\ref{R_HandleInputErrors}, 
						 R\ref{R_DisplayErrorMsg}\\ \hline
		T\ref{NoLCs}& R\ref{R_Inputs}, R\ref{R_HandleInputErrors}, 
					  R\ref{R_DisplayErrorMsg}\\ \hline
		T\ref{Usability}& NFR\ref{NFR_usability}\\ \hline
		T\ref{Portability}& NFR\ref{NFR_portability}\\ \hline
		T\ref{Accuracy}& NFR\ref{NFR_accuracy}\\ \hline
		T\ref{Correctness}& NFR\ref{NFR_correctness}\\ \hline
		T\ref{Performance}& NFR\ref{NFR_performance}\\ \hline
		T\ref{Stability}& NFR\ref{NFR_stability}\\ \hline
	\end{tabular}
	\caption{Traceability Between Test Cases and Requirements}
	\label{Table:Traceability} 
\end{table}
		
\section{Trace to Modules} \label{ModTrace}	

Table \ref{Table:Traceability2} shows the traceability between the test cases 
and the modules. \\

\begin{table} [h!]
	\centering
	\begin{tabular}{|c|c|}
		\hline	
		\textbf{T} & \textbf{Modules}\\
		\hline 
		T\ref{Max_Opt}& Simplex Method Solver Module\\ \hline
		T\ref{Max_NoOpt}& Simplex Method Solver Module, Exceptions Module\\ 
		\hline
		T\ref{Min_Opt}& Simplex Method Solver Module\\ \hline
		T\ref{Min_NoOpt}& Simplex Method Solver Module, Exceptions Module\\ 
		\hline
		T\ref{NoObjcFunc}& Simplex Method Solver Module\\ \hline
		T\ref{NoLCs}& Simplex Method Solver Module\\ \hline
		T\ref{Usability}& Simplex Method Solver Module\\ \hline
		T\ref{Portability}& Simplex Method Solver Module\\ \hline
		T\ref{Accuracy}& Simplex Method Solver Module\\ \hline
		T\ref{Correctness}& Simplex Method Solver Module\\ \hline
		T\ref{Performance}& Simplex Method Solver Module\\ \hline
		T\ref{Stability}& Simplex Method Solver Module\\ \hline
	\end{tabular}
	\caption{Traceability Between Test Cases and Modules}
	\label{Table:Traceability2} 
\end{table}

\newpage

\section{Code Coverage Metrics} \label{CodeCoverage}

This section is discussed in details in the Unit V\&V Report found at: 
\url{https://github.com/hananezlitni/HZ-CAS741-Project/tree/master/docs/VnVReport/UnitVnVReport/UnitVnVReport.pdf}.

\newpage

\bibliographystyle{plainnat}
\bibliography {../../../refs/References}

\newpage

\section{Appendix}

This section provides additional content related to this document.

\subsection{Usability Survey Questions} \label{UsabilityTestingQs}

\subsubsection{Participtant 1: Soumaya Zlitni}

\begin{enumerate}
	\item Did \famname{} successfully provide all the services you requested?
	
	(	\hl{Yes}	/	No	)
	
	Why have you chosen the above response? It was clear what the input needed 
	to be.
	
	\item How confident are you that \famname{} provided you with the correct 
	results?
	
	(	1	/	2	/	3	/	4	/	\hl{5}	) ; \textit{(1) being not 
	confident at all and (5) being very confident}
	
	Why have you given the above score? Because everything was clear and left 
	little room for any confusion.
	
	\item How satisfied are you with the library's response time?  
	
	(	1	/	2	/	3	/	4	/	\hl{5}	) ; \textit{(1) being not 
	satisfied at all and (5) being very satisfied} 
	
	Why have you given the above score? The tool was very fast in giving me the 
	output.
	
	\item How likely are you to recommend \famname{} to a friend?
	
	(	1	/	2	/	3	/	4	/	\hl{5}	) ; \textit{(1) being very 
	unlikely and (5) being very likely} 
	
	Why have you given the above score? The tool does what it is supposed to do.
	
	\item Rate your overall satisfaction with \famname{} out of 10 
	
	(	1	/	2	/	3	/	4	/	5	/	6	/	7	/	8	/	\hl{9}	
	/	10	)
	
	Why have you given the above score? The tool did what it is supposed to do 
	in a very short amount of time. The inputs that should be entered were 
	clear.
	\end {enumerate}

\end{document}