\documentclass{article}

\usepackage{tabularx}
\usepackage{booktabs}

\title{CAS 741: Problem Statement\\A Library of Simplex Method Solvers}

\author{Hanane Zlitni (zlitnih)}

\date{September 14, 2018}

%% Comments

\usepackage{color}

\newif\ifcomments\commentstrue

\ifcomments
\newcommand{\authornote}[3]{\textcolor{#1}{[#3 ---#2]}}
\newcommand{\todo}[1]{\textcolor{red}{[TODO: #1]}}
\else
\newcommand{\authornote}[3]{}
\newcommand{\todo}[1]{}
\fi

\newcommand{\wss}[1]{\authornote{blue}{SS}{#1}}
\newcommand{\hz}[1]{\authornote{magenta}{Author}{#1}}


\begin{document}

\maketitle

\begin{table}[hp]
\caption{Revision History} \label{TblRevisionHistory}
\begin{tabularx}{\textwidth}{llX}
\toprule
\textbf{Date} & \textbf{Developer(s)} & \textbf{Change}\\
\midrule
September 14 & Hanane Zlitni & First Draft\\
\bottomrule
\end{tabularx}
\end{table}

The simplex method, a linear programming algorithm, is considered one of the most popular algorithms that has significant influence in the fields of science and engineering 
\cite{simplex-popularity}. \par 

The algorithm can be used in a variety of fields and its goal is to make the most of the available resources to achieve the optimal solution. For example, the simplex method is 
used in the sand casting process to optimize the sand casting parameters to produce the best results \cite{sand-casting}. Moreover, the simplex method was used in chemistry to 
maximize the yield of a chemical reaction \cite{chemistry}. \par

Since the simplex method has various applications in different fields, a software that facilitates solving objective functions using the simplex method for different purposes can be 
useful. Therefore, I propose the development of a library containing simplex method solvers. It would output the optimal solution of the objective function that satisfies its 
constraints and achieves the desired goal (maximization or minimization) given the objective function, the objective function goal (maximization or minimization) and the linear 
constraints that the objective function is subject to. \par 

To use the library, basic knowledge of linear programming is assumed, but no technical background is required. The library will be operable on different platforms, including Mac 
and Windows. \\

%\wss{comment}
%\hz{comment}

\bibliographystyle {plainnat}
\bibliography{../../refs/ProblemStatementRefs}
\end{document}  