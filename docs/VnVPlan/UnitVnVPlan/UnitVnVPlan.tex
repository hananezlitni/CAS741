\documentclass[12pt, titlepage]{article}

\usepackage{amsmath, mathtools}
\usepackage{booktabs}
\usepackage{tabularx}
\usepackage{xr}
\usepackage{hyperref}
\hypersetup{
    colorlinks,
    citecolor=blue,
    filecolor=magenta,
    linkcolor=black,
    urlcolor=cyan
}
\usepackage[round]{natbib}

%% Comments

\usepackage{color}

\newif\ifcomments\commentstrue

\ifcomments
\newcommand{\authornote}[3]{\textcolor{#1}{[#3 ---#2]}}
\newcommand{\todo}[1]{\textcolor{red}{[TODO: #1]}}
\else
\newcommand{\authornote}[3]{}
\newcommand{\todo}[1]{}
\fi

\newcommand{\wss}[1]{\authornote{blue}{SS}{#1}}
\newcommand{\hz}[1]{\authornote{magenta}{Author}{#1}}


\externaldocument{../SystVnVPlan/SystVnVPlan}
\externaldocument{../../Design/MIS/MIS}

\newcounter{testnum} %Test Number
\newcommand{\dthetestnum}{T\thetestnum}
\newcommand{\tref}[1]{T\ref{#1}}
%%\input{../../Common}

\newcommand{\progname}{Library of Simplex Method Solvers}
\newcommand{\famname}{LoSMS}

\begin{document}

\title{Unit Verification and Validation Plan for a \progname{} (\famname{})} 
\author{Hanane Zlitni}
\date{December 3, 2018}
	
\maketitle

\pagenumbering{roman}

\section{Revision History}

\begin{tabularx}{\textwidth}{p{3cm}p{2cm}X}
\toprule {\bf Date} & {\bf Version} & {\bf Notes}\\
\midrule
December 17, 2018 & 2.0 & Final Draft (includes making the document consistent 
with the rest of the deliverables)\\
December 16, 2018 & 1.2 & Applied Dr.~Smith’s Comments\\
December 6, 2018 & 1.1 & Applied Brooks MacLachlan's Comments Posted on GitHub\\
December 3, 2018 & 1.0 & First Draft\\
\bottomrule
\end{tabularx}

~\newpage

\tableofcontents

\listoftables

\newpage

\section{Symbols, Abbreviations and Acronyms}

The following are symbols, abbreviations or acronyms used in this document: \\

\renewcommand{\arraystretch}{1.2}
\begin{tabular}{l l} 
  \toprule		
  \textbf{symbol} & \textbf{description}\\
  \midrule 
  T & Test\\
  $Z$ & Optimal solution(s) of the objective function\\
  MIS & Module Interface Specification\\
  V\&V & Verification and Validation\\
  LCs & Linear Constraints\\
  ObjcFunc & Objective Function\\
  Const & Constants\\
  max & Maximization\\
  min & Minimization\\
  \bottomrule
\end{tabular}\\

\newpage

\pagenumbering{arabic}

This document describes the unit Verification and Validation (V\&V) plan for
the Library of Simplex Method Solvers (\famname{}) tool. It is intended to be a 
refinement of the tool's system V\&V plan by providing test cases based on 
the modules in the library's Module Interface Specification (MIS) document. The 
MIS, along with the full documentation of \famname{}, can be found at: 
\url{https://github.com/hananezlitni/HZ-CAS741-Project}.\\

The unit V\&V plan starts by providing general information about the tool in 
Section \ref{GeneralInfo}. Then, Section \ref{Plan} provides additional details 
about the plan, which include information about the V\&V team, automated 
testing and verification tools and non-testing based verification. This is 
followed by the unit test description in Section \ref{UnitTestDescription}, 
which consists of tests for the library's functional and nonfunctional 
requirements, categorized based on the modules in the MIS, and traceability 
between the test cases and modules.

\section{General Information} \label{GeneralInfo}

\subsection{Purpose}

The software under test, \famname{}, is a general-purpose program family that 
is intended to be used by people from various backgrounds. It facilitates 
obtaining the optimal solution of a linear program using the simplex method. It 
accepts the objective function, the objective function goal (maximization or 
minimization) and the linear constraints and outputs the optimum of the linear 
programming problem.

\subsection{Scope} \label{Scope}

The core module of the library is the simplex solver module. It contains all 
components of the tool from reading the inputs to calculating and obtaining the 
output. Specifically, the function \textit{solveLP()} is the one that will be 
the focus of unit testing because all other functions are going to be called 
during its execution. Therefore, the test cases in this document, and the 
System V\&V Plan (found at: 
\url{https://github.com/hananezlitni/HZ-CAS741-Project/blob/master/docs/VnVPlan/SystVnVPlan/SystVnVPlan.pdf}),
 are selected to cover all possible cases that could happen to 
\textit{solveLP()}. \\

Some test cases are selected to cover the cases of the occurence of exceptions. 
These will help verify the exceptions module along with the solver module. \\

More information about the modules can be found in the MG document at: 
\url{https://github.com/hananezlitni/HZ-CAS741-Project/blob/master/docs/Design/MG/MG.pdf}.

\wss{I suggest you give the specific names of the modules that will be
  verified.  A reference to the MG would also be a good idea.}\hz{done}

\section{Plan} \label{Plan}
	
\subsection{Verification and Validation Team}

The verification and validation team consists of one member: Hanane Zlitni.

\subsection{Automated Testing and Verification Tools}

Unittest, a unit testing framework for Python, will be used for automated 
testing. Code coverage would be verified using Coverage.py.

\subsection{Non-Testing Based Verification}

Not applicable for \famname{}.

\section{Unit Test Description} \label{UnitTestDescription}

The test cases discussed in this section were selected to ensure that the 
library does correctly what it is intended to do while maintaining quality. The 
test cases were derived from the tool's MIS document which can be found at: 
\url{https://github.com/hananezlitni/HZ-CAS741-Project/blob/master/docs/Design/MIS/MIS.pdf}

\subsection{Tests for Functional Requirements}

The following are the test cases related to the tool's functional requirements 
and categorized based on the modules in the MIS document.

\subsubsection{Simplex Solver Module}

The simplex solver module is the main module of the library. It receives the 
inputs from the driver program and performs the calculations necessary to 
obtain the optimal solution of the linear program.\\

Testing for faulty inputs is detailed in the System V\&V Plan 
(T\ref{NoObjcFunc} \& T\ref{NoLCs}) found at: 
\url{https://github.com/hananezlitni/HZ-CAS741-Project/blob/master/docs/VnVPlan/SystVnVPlan/SystVnVPlan.pdf}\\

As previously explained in Section \ref{Scope}, the focus of the test cases in 
this document will be on \textit{solveLP()}. The following test cases were 
selected to test each possible case that can occur in the function execution.

\begin{enumerate}
	
	\item{\textbf{T\refstepcounter{testnum}\thetestnum \label{solveLpMax1}: 
	First Test for solveLP() for maximization problems}}

	Control: Automatic
	
	Initial State: -
	
	Input: LCs = [[20,6,3],[0,1,0],[-1,-1,1],[-9,1,1]], Constants = 
	[182,10,0,0], ObjcFunc = [100000,40000,18000], Goal = max
	
	Output: $Z$ = 1052000, $x_1$ = 4, $x_2$ = 10, $x_3$ = 14
	 	
	Test Case Derivation: Section \ref{M_SimplexSolver} in the MIS 
	(\cite{losms-mis})
	
	How test will be performed: Unittest

	\item{\textbf{T\refstepcounter{testnum}\thetestnum \label{solveLpMax2}: 
			Second Test for solveLP() for maximization problems}}
	
	Control: Automatic
	
	Initial State: -
	
	Input: LCs = [[3,2],[1,0],[0,1],[-9,1,1]], Constants = [182,40,60], 
	ObjcFunc = [3,2], Goal = max
	
	Output: $Z$ = 180, $x_1$ = 40, $x_2$ = 30
	
	Test Case Derivation: Section \ref{M_SimplexSolver} in the MIS 
	(\cite{losms-mis})
	
	How test will be performed: Unittest
	
	\item{\textbf{T\refstepcounter{testnum}\thetestnum \label{solveLpMin}: Test 
	solveLP() for minimization problem with no optimal solution}}
	
	Control: Automatic
	
	Initial State: -
	
	Input: LCs = [[400,300],[300,400],[200,500]], Constants = 
	[25000,27000,30000], ObjcFunc = [11,16,15], Goal = min
	
	Output: Exception message: ``This linear program does not have an optimal 
	solution''
	
	Test Case Derivation: Section \ref{M_SimplexSolver} in the MIS 
	(\cite{losms-mis})
	
	How test will be performed: Unittest
\end{enumerate}

\wss{I like your explanations; they add to the document.  Unfortunately
	they will disappear when the comments are turned off.  I suggest that
	you move this information out of the comment and into the document.  You
	might even keep the heading explanation and add it to your template.}

\subsubsection{Explanation}

The function \textit{solveLP()} receives 4 inputs in the following order:
\begin{enumerate}
	\item The coefficients of the linear constraints (LCs) in the form of a 
	2-dimentional array
	\item The constants (const) in the linear constraints, such that: Ax $\leq$ 
	const
	\item The coefficients of the objective function (ObjcFunc)
	\item The linear program goal (max or min)
\end{enumerate}

Most of the test cases for the library are system tests (detailed in the System 
V\&V Plan). The test cases in this document are added to ensure a high test 
coverage.

\subsection{Tests for Nonfunctional Requirements}

The test cases T\ref{Usability}-T\ref{Stability} in section 
\ref{NonFunctionalTests} in the System V\&V Plan cover the non-functional 
requirements (qualities) that are important for \famname{}. The document can be 
found at: 
\url{https://github.com/hananezlitni/HZ-CAS741-Project/blob/master/docs/VnVPlan/SystVnVPlan/SystVnVPlan.pdf}.

\subsection{Code Coverage}
\hz{Sorry for changing the template.. but I wanted to add a section for code 
	coverage because it is planned for \famname{}.}

It is planned to verify the code coverage that the test cases for \famname{} 
achieves. The aim is to reach a minimum of 99\% code coverage. In the case of a 
lower percentage, more test cases would be added for the parts of the 
implementation that weren't covered. \\

As explained at the beginning of this document, Python's Coverage.py will be 
used to verify the code coverage. It provides a table with the coverage 
percentage along with a detailed view on the parts of the implementation that 
were and weren't covered.

\subsection{Traceability Between Test Cases and Modules}

\begin{table} [h!]
	\centering
	\begin{tabular}{|c|c|}
		\hline	
		\textbf{Test Case Number} & \textbf{Module}\\
		\hline 
		T\ref{NoObjcFunc} \& T\ref{NoLCs} in System V\&V Plan& Simplex Solver 
		Module\\ \hline
		\tref{solveLpMax1}& Simplex Method Solver Module\\ \hline
		\tref{solveLpMax2}& Simplex Method Solver Module\\ \hline
		\tref{solveLpMin}& Simplex Solver Module, Exceptions Module\\ \hline
	\end{tabular}
	\caption{Traceability Between Test Cases and Modules}
	\label{Table:Traceability} 
\end{table}

\wss{test cases look good.}

\newpage

\bibliographystyle{plainnat}
\bibliography {../../../refs/References}

\newpage

\section{Appendix}

This section provides additional content related to this document.

\subsection{Symbolic Parameters}

There are no symbolic parameters used in this document.

\end{document}