\documentclass[12pt, titlepage]{article}

\usepackage{amsmath, mathtools}
\usepackage{booktabs}
\usepackage{tabularx}
\usepackage{xr}
\usepackage{hyperref}
\hypersetup{
    colorlinks,
    citecolor=blue,
    filecolor=magenta,
    linkcolor=black,
    urlcolor=cyan
}
\usepackage[round]{natbib}

%% Comments

\usepackage{color}

\newif\ifcomments\commentstrue

\ifcomments
\newcommand{\authornote}[3]{\textcolor{#1}{[#3 ---#2]}}
\newcommand{\todo}[1]{\textcolor{red}{[TODO: #1]}}
\else
\newcommand{\authornote}[3]{}
\newcommand{\todo}[1]{}
\fi

\newcommand{\wss}[1]{\authornote{blue}{SS}{#1}}
\newcommand{\hz}[1]{\authornote{magenta}{Author}{#1}}

\externaldocument{../SystVnVPlan/SystVnVPlan}
\externaldocument{../../Design/MIS/MIS}
%%\input{../../Common}

\newcommand{\progname}{Library of Simplex Method Solvers}
\newcommand{\famname}{LoSMS}

\begin{document}

\title{\famname{}: Unit Verification and Validation Plan for a \progname{}} 
\author{Hanane Zlitni}
\date{December 3, 2018}
	
\maketitle

\pagenumbering{roman}

\section{Revision History}

\begin{tabularx}{\textwidth}{p{3cm}p{2cm}X}
\toprule {\bf Date} & {\bf Version} & {\bf Notes}\\
\midrule
December 6, 2018 & 1.1 & Applied Brooks MacLachlan's Comments Posted on GitHub\\
December 3, 2018 & 1.0 & First Draft\\
\bottomrule
\end{tabularx}

~\newpage

\tableofcontents

\listoftables

\newpage

\section{Symbols, Abbreviations and Acronyms}

The following are symbols, abbreviations or acronyms used in this document: \\

\renewcommand{\arraystretch}{1.2}
\begin{tabular}{l l} 
  \toprule		
  \textbf{symbol} & \textbf{description}\\
  \midrule 
  T & Test\\
  $Z$ & Optimal solution(s) of the objective function\\
  $K$ & The points where the optimal solution(s) occur\\
  MIS & Module Interface Specification\\
  V\&V & Verification and Validation\\
  \bottomrule
\end{tabular}\\

\newpage

\pagenumbering{arabic}

This document describes the unit Verification and Validation (V\&V) plan for
the Library of Simplex Method Solvers (\famname{}) tool. It is intended to be a 
refinement of the tool's system V\&V plan by providing test cases based on 
the modules in the library's Module Interface Specification (MIS) document. The 
MIS, along with the full documentation of \famname{}, can be found at: 
\url{https://github.com/hananezlitni/HZ-CAS741-Project}.

The unit V\&V plan starts by providing general information about the tool in 
Section \ref{GeneralInfo}. Then, Section \ref{Plan} provides additional details 
about the plan, which include information about the V\&V team, automated 
testing and verification tools and non-testing based verification. This is 
followed by the unit test description in Section \ref{UnitTestDescription}, 
which consists of tests for the library's functional and nonfunctional 
requirements, categorized based on the modules in the MIS, and traceability 
between the test cases and modules.

\section{General Information} \label{GeneralInfo}

\subsection{Purpose}

The software under test, \famname{}, is a general-purpose program family that 
is intended to be used by people from various backgrounds. It facilitates 
obtaining the optimal solution of a linear program using the simplex method. It 
accepts the objective function, the objective function goal (maximization or 
minimization) and the linear constraints and outputs the optimum of the linear 
programming problem.

\subsection{Scope}

All the modules, except for the output module, will be verified using the test 
cases in this document and the System V\&V Plan (found at: 
\url{https://github.com/hananezlitni/HZ-CAS741-Project/blob/master/docs/VnVPlan/SystVnVPlan/SystVnVPlan.pdf}).
 The reason for excluding the output module is because its sole responsibility 
 is to display the solution and there are no calculations included in it. 
 Therefore, priority was given for the other modules.

\section{Plan} \label{Plan}
	
\subsection{Verification and Validation Team}

The verification and validation team consists of one member: Hanane Zlitni.

\subsection{Automated Testing and Verification Tools}

PyTest, a unit testing framework for Python, will be used for automated 
testing. Code coverage would be achieved by selecting test cases that cover 
every possible statement and branch. In addition, Python's Mock library will be 
used to replace parts of the modules with mock objects and check their 
behaviour.

\subsection{Non-Testing Based Verification}

Not applicable for \famname{}.

\section{Unit Test Description} \label{UnitTestDescription}

The test cases discussed in this section were selected to ensure that the 
library does correctly what it is intended to do while maintaining quality. The 
test cases are derived from the tool's MIS document which can be found at: 
\url{https://github.com/hananezlitni/HZ-CAS741-Project/blob/master/docs/Design/MIS/MIS.pdf}

\subsection{Tests for Functional Requirements}

The following are the test cases related to the tool's functional requirements 
and categorized based on the modules in the MIS document.

\subsubsection{Input Module}

The input module is responsible for receiving the necessary inputs, verifying 
them and throwing an exception in case of a violation.

The test cases T6-T9 in \ref{TestInputs} in the System V\&V Plan cover the 
cases related to the input module. The document can be found at: 
\url{https://github.com/hananezlitni/HZ-CAS741-Project/blob/master/docs/VnVPlan/SystVnVPlan/SystVnVPlan.pdf}

\subsubsection{Tableau Module}

The tableau module is responsible for operations related to the simplex 
tableau, such as setting it up, updating its values and converting the tableau 
to the canonical form. 

Since the conversion includes the execution of most of the module's access 
programs (found in the MIS document \cite{losms-mis}), the following test cases 
were selected to test each possible case in toCanonical() - which helps 
achieving branch coverage.

In addition, a test case was selected for updateTableau() to cover the case 
when negating the objective function is needed.

\begin{enumerate}
		
	\item{\textbf{T1: Test canonical form conversion for maximization problems 
	with less than or equal to inequalities}}
		
	Control: Automatic 
		
	Initial State: sTableau = $\begin{bmatrix}
	-2 & 3 & -1 & 1 & 0\\
	1 & 1 & 1 & 0 & 10\\
	4 & -3 & 1 & 0 & 3\\
	2 & 1 & -1 & 0 & 10\\
	\end{bmatrix}$
		
	Input: -
		
	Output: No output, but the state changes to:
	
	sTableau = $\begin{bmatrix}
	-2 & 3 & -1 & 1 & 0 & 0 & 0 & 0\\
	1 & 1 & 1 & 0 & 1 & 0 & 0 & 10\\
	4 & -3 & 1 & 0 & 0 & 1 & 0 & 3\\
	2 & 1 & -1 & 0 & 0 & 0 & 1 & 10\\
	\end{bmatrix}$
	
	Test Case Derivation: Section \ref{M_Tableau} in the MIS (\cite{losms-mis})
		
	How test will be performed: PyTest and/or Mock
	
	\hz{Explanation: Since the linear constraints are less than or equal to 
	(from calling getLCsType()), the slack variables are added to the matrix} \\

	\item{\textbf{T2: Test canonical form conversion for minimization problems 
	with less than or equal to inequalities}}
	
	Control: Automatic 
	
	Initial State: sTableau = $\begin{bmatrix}
	2 & -3 & 1 & 0\\
	3 & 4 & 0 & 24\\
	7 & 4 & 0 & 16\\
	\end{bmatrix}$ , wasMin = False
	
	Input: -
	
	Output: No output, but the state changes to:
	
	sTableau = $\begin{bmatrix}
	2 & -3 & 1 & 0 & 0 & 0\\
	3 & 4 & 0 & 1 & 0 & 24\\
	7 & 4 & 0 & 0 & 1 & 16\\
	\end{bmatrix}$ , wasMin = True
	
	Test Case Derivation: Section \ref{M_Tableau} in the MIS (\cite{losms-mis})
	
	How test will be performed: PyTest and/or Mock
	
	\item{\textbf{T3: Test updateTableau() for negating the objective function}}
	
	Control: Automatic 
	
	Initial State: sTableau = $\begin{bmatrix}
	2 & -3 & 1 & 0\\
	3 & 4 & 0 & 24\\
	7 & 4 & 0 & 16\\
	\end{bmatrix}$
	
	Input: 0
	
	Output: No output, but the state changes to:
	
	sTableau = $\begin{bmatrix}
	-2 & 3 & -1 & 0 & 0 & 0\\
	3 & 4 & 0 & 1 & 0 & 24\\
	7 & 4 & 0 & 0 & 1 & 16\\
	\end{bmatrix}$
	
	Test Case Derivation: Section \ref{M_Tableau} in the MIS (\cite{losms-mis})
	
	How test will be performed: PyTest and/or Mock
	
	\hz{Explanation: T1 and T2 are also tests for updateTableau(), since the 
	conversion to the canonical form cannot be done without updating the 
	tableau. Since they don't cover if updateTableau() was called for 
	converting maximization to minimization (negating the first row in the 
	tableau), I added T3 to test that case.}

\end{enumerate}

\subsubsection{Simplex Solver Module}

The simplex solver module is responsible for performing the simplex method 
steps to calculate the optimum and the points where it occurs.

The following test cases were selected to cover the cases for solving 
minimization and maximization problems.

\begin{enumerate}
	
	\item{\textbf{T4: Test solveLP() for minimization problems}}

	Control: Automatic
	
	Initial State: Z = [ ] , K = [ ]
	
	Input: 	$\begin{bmatrix}
	2 & -3 & 1 & 0 & 0 & 0\\
	3 & 4 & 0 & 1 & 0 & 24\\
	7 & 4 & 0 & 0 & 1 & 16\\
	\end{bmatrix}$ , True 
	
	Output: No output, but the state changes to:
	
	Z = [-4.57] , K = [2.29 , 0]
	
	Test Case Derivation: Section \ref{M_simplexSolver} in the MIS 
	(\cite{losms-mis})
	
	How test will be performed: PyTest and/or Mock
	
	\item{\textbf{T5: Test solveLP() for maximization problems}}
	
	Control: Automatic
	
	Initial State: Z = [ ] , K = [ ]
	
	Input: 	$\begin{bmatrix}
	-2 & 3 & -1 & 1 & 0 & 0 & 0 & 0\\
	1 & 1 & 1 & 0 & 1 & 0 & 0 & 10\\
	4 & -3 & 1 & 0 & 0 & 1 & 0 & 3\\
	2 & 1 & -1 & 0 & 0 & 0 & 1 & 10\\
	\end{bmatrix}$ , False 
	
	Output: No output, but the state changes to:
	
	Z = [3] , K = [0, 0, 3]
	
	Test Case Derivation: Section \ref{M_simplexSolver} in the MIS 
	(\cite{losms-mis})
	
	How test will be performed: PyTest and/or Mock
		
\end{enumerate}

\subsection{Tests for Nonfunctional Requirements}

The test cases T11-T15 in section \ref{NonFunctionalTests} in the System V\&V 
Plan cover the non-functional requirements (qualities) that are important for 
\famname{}. The document can be found at: 
\url{https://github.com/hananezlitni/HZ-CAS741-Project/blob/master/docs/VnVPlan/SystVnVPlan/SystVnVPlan.pdf}

\subsection{Traceability Between Test Cases and Modules}

\begin{table} [h!]
	\centering
	\begin{tabular}{|c|c|}
		\hline	
		\textbf{Test Case Number} & \textbf{Module}\\
		\hline 
		T6-T9 in System V\&V Plan& Input Module\\ \hline
		T1& Tableau Module\\ \hline
		T2& Tableau Module\\ \hline
		T3& Tableau Module\\ \hline
		T4& Simplex Solver Module\\ \hline
		T5& Simplex Solver Module\\ \hline
	\end{tabular}
	\caption{Traceability Between Test Cases and Modules}
	\label{Table:Traceability} 
\end{table}

\newpage

\bibliographystyle{plainnat}
\bibliography {../../../refs/References}

\newpage

\section{Appendix}

This section provides additional content related to this document.

\subsection{Symbolic Parameters}

There are no symbolic parameters used in this document.

\end{document}