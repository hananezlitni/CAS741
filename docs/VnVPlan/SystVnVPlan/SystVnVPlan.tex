\documentclass[12pt, titlepage]{article}

\usepackage{booktabs}
\usepackage{tabularx}
\usepackage{hyperref}
\hypersetup{
    colorlinks,
    citecolor=blue,
    filecolor=magenta,
    linkcolor=black,
    urlcolor=cyan
}
\usepackage[round]{natbib}



%% Comments

\usepackage{color}

\newif\ifcomments\commentstrue

\ifcomments
\newcommand{\authornote}[3]{\textcolor{#1}{[#3 ---#2]}}
\newcommand{\todo}[1]{\textcolor{red}{[TODO: #1]}}
\else
\newcommand{\authornote}[3]{}
\newcommand{\todo}[1]{}
\fi

\newcommand{\wss}[1]{\authornote{blue}{SS}{#1}}
\newcommand{\hz}[1]{\authornote{magenta}{Author}{#1}}


\newcommand{\famname}{LoSMS} % PUT YOUR PROGRAM NAME %HERE

\begin{document}

\title{A Library of Simplex Method Solvers: System Verification and Validation 
Plan} 
\author{Hanane Zlitni}
\date{October 22, 2018}
	
\maketitle

\pagenumbering{roman}

\section{Revision History}

\begin{tabularx}{\textwidth}{p{3cm}p{2cm}X}
\toprule {\bf Date} & {\bf Version} & {\bf Notes}\\
\midrule
October 22, 2018 & 1.0 & First Draft\\
\bottomrule
\end{tabularx}

~\newpage

\section{Symbols, Abbreviations and Acronyms}

\renewcommand{\arraystretch}{1.2}
\begin{tabular}{l l} 
  \toprule		
  \textbf{symbol} & \textbf{description}\\
  \midrule 
  T & Test\\
  V\&V & Verification and Validation\\
  \famname{} & Library of Simplex Method Solvers\\
  CA & Commonality Analysis\\
  SRS & Software Requirements Specification\\
  A & Assumption\\
  IM & Instance Model\\
  $s.\;t.$ & Subject to\\
  $Z$ & Optimal solution of the objective function\\
  $x_1, x_2, x_3$ & Decision variables\\
  \bottomrule
\end{tabular}\\

\newpage

\tableofcontents

\listoftables

\listoffigures

\newpage

\pagenumbering{arabic}

This document describes the system verification and validation (V\&V) plan for 
the Library of Simplex Method Solvers (\famname{}) tool. It is based on the 
tool's commonality analysis (CA) that can be found, along with the full 
documentation of \famname{}, in the following link:  
\url{https://github.com/hananezlitni/HZ-CAS741-Project}.

The V\&V plan starts by providing general information about the tool and this 
document in Section \ref{GeneralInfo}. Then, Section \ref{Plan} provides 
additional details about the plan, which includes information about the V\&V 
team, the SRS, design and implementation verification plans and the software 
validation plan. This is followed by the system test description in Section 
\ref{SystemTestDescription}, which consists of tests for the tool's functional 
and nonfunctional requirements and traceability between test cases and 
requirements. The document is concluded by Section \ref{StaticVerTechniques} 
which describes the techniques for static verification.

\section{General Information} \label{GeneralInfo}

\subsection{Summary}

The software under test, \famname{}, is a general-purpose program family that 
facilitates obtaining the optimal solution of a linear program, using the 
simplex method, given the objective function, the objective function goal 
(maximization or minimization) and the linear constraints. Since the simplex 
algorithm is widely used in various fields, \famname{} is intended to be used 
by people from different backgrounds to help them optimize parameters of their 
choice.

\subsection{Objectives}

The objective of this verification and validation plan is to build confidence 
in the correctness of the \famname{} tool (i.e. it produces the correct output 
for the corresponding inputs), while providing satisfactory usability.

\subsection{References}

Different sections in this document refer to the tool's CA (\cite{losms-ca}).

\section{Plan} \label{Plan}
	
\subsection{Verification and Validation Team}

The verification and validation team consists of one member: Hanane Zlitni.

\subsection{SRS Verification Plan}

The CA for the \famname{} tool will be verified by getting feedback from Dr. 
Spencer Smith and my CAS 741 classmates which comes from formal reviews of the 
document. 

The feedback from Dr. Smith is expressed as comments written in the actual 
document. The feedback from the CAS 741 students, on the other hand, is 
expressed as issues posted in the project's GitHub repository. After getting 
the comments, I apply them one by one and make changes in the document.

\subsection{Design Verification Plan}

\famname{}'s design documents will be verified by getting feedback from 
Dr. Spencer Smith and my CAS 741 classmates which comes from formal reviews of 
the documents.

The feedback from Dr. Smith is expressed as comments written in the documents, 
whereas the feedback from the CAS 741 students is expressed as issues posted in 
the project's GitHub repository. After getting the comments, I apply them one 
by one and make changes in the documents.

\subsection{Implementation Verification Plan}

The implementation of the \famname{} tool will be verified statically by 
performing code review with Dr. Spencer Smith and my CAS 741 classmates (and 
getting the comments in the form GitHub issues) and dynamically by executing 
the test cases detailed in this plan and the unit V\&V plan using testing 
frameworks (e.g. JUnit/PyUnit). Based on the comments/errors I get, the 
implementation will be modified.
 
\subsection{Software Validation Plan}

Not applicable for \famname{}.

\section{System Test Description} \label{SystemTestDescription}

System testing for the \famname{} tool ensures that the correct inputs produce 
the correct outputs. The test cases in this section are derived from the 
instance models and the requirements detailed in the tool's CA.
	
\subsection{Tests for Functional Requirements}

\subsubsection{Tests for Solving Maximization Linear Programs}

\begin{enumerate}
	\item{\textbf{T1: Unique Optimal Solution}}
	
	Control: Automatic 
	
	Initial State: -
	
	Input: $max\;Z\;=\;2x_1\;-\;3x_2\;+\;x_3$\\
	$s.\;t.$$\hspace*{1.3cm} x_1\;+\;x_2\;+\;x_3\; \leq \;10$\\
	$\hspace*{2cm} 4x_1\;-\;3x_2\;+\;x_3\; \leq \;3$\\
	$\hspace*{2cm} 2x_1\;+\;x_2\;-\;x_3\; \leq \;10$\\
	$\hspace*{2cm} x_1\;,\;x_2\;,\;x_3\; \geq \;0$
	
	Output: $Z = 3$, occurring when $x_1=0,\;x_2=0,\;x_3=3$
	
	How test will be performed: Unit testing using JUnit/PyUnit
	
	Test Case Derivation: IM1 in \cite{losms-ca}

	\item{\textbf{T2: Multiple Optimal Solutions}}
	
	Control: Automatic 
	
	Initial State: -
	
	Input: $max\;Z\;=\;3x_1\;+\;2x_2$\\
	$s.\;t.$$\hspace*{1.3cm} 3x_1\;+\;2x_2\; \leq \;180$\\
	$\hspace*{2cm} x_1\; \leq \;40$\\
	$\hspace*{2cm} x_2\; \leq \;60$\\
	$\hspace*{2cm} x_1\;,\;x_2\; \geq \;0$
	
	Output: $Z = 180$, occurring when $x_1=40,\;x_2=30$ \& $x_1=20,\;x_2=60$
	
	How test will be performed: Unit testing using JUnit/PyUnit
	
	Test Case Derivation: IM1 in \cite{losms-ca}

	\item{\textbf{T3: No Optimal Solution}}
	
	Control: Automatic 
	
	Initial State: -
	
	Input: $max\;Z\;=\;2x_1\;+\;x_2$\\
	$s.\;t.$$\hspace*{1.3cm} x_1\;-\;x_2\; \leq \;10$\\
	$\hspace*{2cm} 2x_1\;-\;x_2\; \leq \;40$\\
	$\hspace*{2cm} x_1\;,\;x_2\; \geq \;0$
	
	Output: ``This linear program does not have an optimal solution'', or a 
	corresponding exception
	
	How test will be performed: Unit testing using JUnit/PyUnit
	
	Test Case Derivation: IM1 in \cite{losms-ca} and 
	\cite{simplex-special-situations}
\end{enumerate}

\subsubsection{Tests for Solving Minimization Linear Programs}

\begin{enumerate}
	\item{\textbf{T4: Unique Optimal Solution}}
	
	Control: Automatic 
	
	Initial State: -
	
	Input: $min\;Z\;=\;-2x_1\;+\;3x_2$\\
	$s.\;t.$$\hspace*{1.3cm} 3x_1\;+\;4x_2\; \leq \;24$\\
	$\hspace*{2cm} 7x_1\;+\;4x_2\; \leq \;16$\\
	$\hspace*{2cm} x_1\;,\;x_2\; \geq \;0$
	
	Output: $Z = -4.57$, occurring when $x_1=2.29,\;x_2=0$
	
	How test will be performed: Unit testing using JUnit/PyUnit
	
	Test Case Derivation: IM2 in \cite{losms-ca}
	
	\item{\textbf{T5: No Optimal Solution}}
	
	Control: Automatic 
	
	Initial State: -
	
	Input: $min\;Z\;=\;3x_1\;+\;14x_2$\\
	$s.\;t.$$\hspace*{1.3cm} -x_1\;-\;5x_2\; \leq \;-6$\\
	$\hspace*{2cm} -x_1\;-\;4x_2\; \leq \;-5$\\
	$\hspace*{2cm} -x_1\;-\;3x_2\; \leq \;-4$\\
	$\hspace*{2cm} -x_1\;-\;2x_2\; \leq \;-5$\\
	$\hspace*{2cm} -x_1\;-\;x_2\; \leq \;-6$\\
	$\hspace*{2cm} x_1\;,\;x_2\; \geq \;0$
	
	Output: ``This linear program does not have an optimal solution'', or a 
	corresponding exception
	
	How test will be performed: Unit testing using JUnit/PyUnit
	
	Test Case Derivation: IM2 in \cite{losms-ca}
\end{enumerate}

\subsubsection{Tests for Faulty Inputs}

\begin{enumerate}
	\item{\textbf{T6: No Objective Function}}
	
	Control: Automatic
	
	Initial State: -
	
	Input: $min $\\
	$s.\;t.$$\hspace*{1.3cm} x_1\;+\;x_2\;+\;x_3\; \leq \;10$\\
	$\hspace*{2cm} 4x_1\;-\;3x_2\;+\;x_3\; \leq \;3$\\
	$\hspace*{2cm} 2x_1\;+\;x_2\;-\;x_3\; \leq \;10$\\
	$\hspace*{2cm} x_1\;,\;x_2\;,\;x_3\; \geq \;0$
	
	Output: ``Error: No objective function'', or a corresponding exception
	
	How test will be performed: Unit testing using JUnit/PyUnit
	
	Test Case Derivation: IM1 \& IM2 in \cite{losms-ca}
	
	\item{\textbf{T7: No Linear Constraints}}
	
	Control: Automatic
	
	Initial State: -
	
	Input: $max\;Z\;=\;2x_1\;-\;3x_2\;+\;x_3$
	
	Output: ``Error: No linear constraints'', or a corresponding exception
	
	How test will be performed: Unit testing using JUnit/PyUnit
	
	Test Case Derivation: IM1 \& IM2 in \cite{losms-ca}
	
	\item{\textbf{T8: No Objective Function Goal}}
	
	Control: Automatic
	
	Initial State: -
	
	Input: $\;Z\;=\;2x_1\;-\;3x_2\;+\;x_3$\\
	$s.\;t.$$\hspace*{1.3cm} x_1\;+\;x_2\;+\;x_3\; \leq \;10$\\
	$\hspace*{2cm} 4x_1\;-\;3x_2\;+\;x_3\; \leq \;3$\\
	$\hspace*{2cm} 2x_1\;+\;x_2\;-\;x_3\; \leq \;10$\\
	$\hspace*{2cm} x_1\;,\;x_2\;,\;x_3\; \geq \;0$
	
	Output: ``Error: No objective function goal'', or a corresponding exception
	
	How test will be performed: Unit testing using JUnit/PyUnit
	
	Test Case Derivation: IM1 \& IM2 in \cite{losms-ca}
	
	\item{\textbf{T9: No Non-negativity Constraints/Negative Decision 
	Variables}}
	
	Control: Automatic
	
	Initial State: -
	
	Input: $max\;Z\;=\;2x_1\;-\;3x_2$\\
	$s.\;t.$$\hspace*{1.3cm} x_1\;+\;x_2\; \leq \;5$\\
	$\hspace*{2cm} 4x_1\;-\;3x_2\; \leq \;4$
	
	Output: ``Error: The decision variables must be positive'', or a 
	corresponding exception
	
	How test will be performed: Unit testing using JUnit/PyUnit
	
	Test Case Derivation: Theoretical Model 1 in \cite{losms-ca}
	
	\item{\textbf{T10: Greater Than or Equal to Inequalities}}
	
	Control: Automatic
	
	Initial State: -
	
	Input: $max\;Z\;=\;2x_1\;-\;3x_2$\\
	$s.\;t.$$\hspace*{1.3cm} x_1\;+\;x_2\; \leq \;5$\\
	$\hspace*{2cm} 4x_1\;-\;3x_2\; \geq \;4$\\
	$\hspace*{2cm} x_1\;,\;x_2\; \geq \;0$
	
	Output: ``Error: The inequalities of the main constraints must be of type 
	less than or equal to'', or a corresponding exception
	
	How test will be performed: Unit testing using JUnit/PyUnit
	
	Test Case Derivation: A2 in \cite{losms-ca}
\end{enumerate}

\subsection{Tests for Nonfunctional Requirements}

\subsubsection{Usability}

\begin{enumerate}
	\item{\textbf{T11: Test for the Usability of \famname{}}}
	
	Type: Usability Testing
						
	Initial State: -
						
	Input/Condition: -
						
	Output/Result: -
						
	How test will be performed: Asking participants to try the tool then answer 
	the usability survey questions (see Appendix \ref{UsabilityTesting}). The 
	goal is to ensure that the library provided the services the users 
	requested and that they are satisfied with the results obtained.
\end{enumerate}

\subsubsection{Portability}

\begin{enumerate}
	\item{\textbf{T12: Test for the Portability of \famname{}}}
	
	Type: Static
	
	Initial State: -
	
	Input/Condition: -
	
	Output/Result: -
	
	How test will be performed: Running \famname{} on Mac, Windows and Linux 
	operating systems
\end{enumerate}

\subsubsection{Accuracy}

\begin{enumerate}
	\item{\textbf{T13: Test for the Accuracy of the Outputs}}

	Type: Dynamic
	
	Initial State: -
	
	Input/Condition: -
	
	Output/Result: -
	
	How test will be performed: I plan to report the relative error of the 
	expected output for each test case detailed in this document
\end{enumerate}

\subsubsection{Correctness \& Performance}

\begin{enumerate}
	\item{\textbf{T14: Test for the Correctness \& Performance of \famname{}}}
	
	Type: Parallel Testing
	
	Initial State: -
	
	Input/Condition: -
	
	Output/Result: -
	
	How test will be performed: I plan to make a comparison between \famname{} 
	and MatLab to evaluate the correctness and performance of \famname{}
\end{enumerate}

\subsubsection{Stability}

\begin{enumerate}
	\item{\textbf{T15: Test for the Stability of \famname{} Under Heavy Load}}
	
	Type: Stress Testing
	
	Initial State: -
	
	Input/Condition: -
	
	Output/Result: -
	
	How test will be performed: Use the library to solve problems with great 
	number of inputs and observe how it would behave
\end{enumerate}

\subsection{Traceability Between Test Cases and Requirements}

The following table describes the mapping between the test cases and 
requirements.

\begin{table} [h!]
	\centering
	\begin{tabular}{|c|c|}
		\hline	
		\textbf{Test Case Number} & \textbf{Requirements}\\
		\hline 
		T1& R1, R4, R5, R6, R7\\ \hline
		T2& R1, R4, R5, R6, R7\\ \hline
		T3& R1, R4, R5, R6, R8\\ \hline
		T4& R1, R4, R5, R6, R7\\ \hline
		T5& R1, R4, R5, R6, R8\\ \hline
		T6& R1, R2, R3\\ \hline
		T7& R1, R2, R3\\ \hline
		T8& R1, R2, R3\\ \hline
		T9& R1, R2, R3\\ \hline
		T10& R1, R2, R3\\ \hline
	\end{tabular}
	\caption{Traceability Between Test Cases and Requirements}
	\label{Table:Traceability} 
\end{table}

\section{Static Verification Techniques} \label{StaticVerTechniques}

Static verification of the \famname{} library implementation will performed 
using code review with Dr. Spencer Smith and my CAS 741 classmates.

\newpage
				
\bibliographystyle {plainnat}
\bibliography {../../../refs/References}

\newpage

\section{Appendix}

This section provides additional content related to this system V\&V plan.

\subsection{Symbolic Parameters}

There are no symbolic parameters used in this document.

\subsection{Usability Survey Questions} \label{UsabilityTesting}

\begin{enumerate}
	\item Did \famname{} successfully provide all the services you requested?
	
	(	Yes	/	No	)
	
	Why have you chosen the above response?
	
	\item How confident are you that \famname{} provided you with the correct 
	results?
	
	(	1	/	2	/	3	/	4	/	5	) ; \textit{(1) being not confident 
	at all and (5) being very confident}

	Why have you given the above score?

	\item How satisfied are you with the library's response time?  
	
	(	1	/	2	/	3	/	4	/	5	) ; \textit{(1) being not satisfied 
	at all and (5) being very satisfied}

	Why have you given the above score?

	\item How likely are you to recommend \famname{} to a friend?
	
	(	1	/	2	/	3	/	4	/	5	) ; \textit{(1) being very unlikely 
	and (5) being very likely}

	Why have you given the above score?

	\item Rate your overall satisfaction with \famname{} out of 10 
	
	(	1	/	2	/	3	/	4	/	5	/	6	/	7	/	8	/	9	/	
	10	)
	
	Why have you given the above score?
\end {enumerate}

\end{document}