\documentclass[12pt, titlepage]{article}

\usepackage{amsmath, mathtools}

\usepackage[round]{natbib}
\usepackage{amsfonts}
\usepackage{amssymb}
\usepackage{graphicx}
\usepackage{colortbl}
\usepackage{xcolor}
\usepackage{xr}
\usepackage{hyperref}
\usepackage{longtable}
\usepackage{xfrac}
\usepackage{tabularx}
\usepackage{float}
\usepackage{siunitx}
\usepackage{booktabs}
\usepackage{multirow}
\usepackage[section]{placeins}
\usepackage{caption}
\usepackage{fullpage}

\hypersetup{
bookmarks=true,     % show bookmarks bar?
colorlinks=true,       % false: boxed links; true: colored links
linkcolor=black,          % color of internal links (change box color with 
%linkbordercolor)
citecolor=blue,      % color of links to bibliography
filecolor=magenta,  % color of file links
urlcolor=cyan          % color of external links
}

\usepackage{array}

\externaldocument{../../SRS/CA}

%% Comments

\usepackage{color}

\newif\ifcomments\commentstrue

\ifcomments
\newcommand{\authornote}[3]{\textcolor{#1}{[#3 ---#2]}}
\newcommand{\todo}[1]{\textcolor{red}{[TODO: #1]}}
\else
\newcommand{\authornote}[3]{}
\newcommand{\todo}[1]{}
\fi

\newcommand{\wss}[1]{\authornote{blue}{SS}{#1}}
\newcommand{\hz}[1]{\authornote{magenta}{Author}{#1}}


\newcommand{\progname}{Library of Simplex Method Solvers}
\newcommand{\famname}{LoSMS}
\definecolor{pyComment}{RGB}{179, 179, 204}

\begin{document}

\title{Module Interface Specification for a \progname{} (\famname{})}

\author{Hanane Zlitni}

\date{November 24, 2018}

\maketitle

\pagenumbering{roman}

\section{Revision History}

\begin{tabularx}{\textwidth}{p{3cm}p{2cm}X}
\toprule {\bf Date} & {\bf Version} & {\bf Notes}\\
\midrule
December 03, 2018 & 1.1 & Correction of solveLP() access program definition in 
8.4.4\\
November 24, 2018 & 1.0 & First draft\\
\bottomrule
\end{tabularx}

~\newpage

\section{Symbols, Abbreviations and Acronyms}

See SRS Documentation at 
\url{https://github.com/hananezlitni/HZ-CAS741-Project/blob/master/docs/SRS/CA.pdf}.
 \\

The following are additional symbols, abbreviations or acronyms used in this 
document: \\

\renewcommand{\arraystretch}{1.2}
\begin{tabular}{l l} 
	\toprule		
	\textbf{symbol} & \textbf{description}\\
	\midrule 
	$Z$ & Optimal solution(s) of the objective function\\
	$K$ & The points where the optimal solution(s) occur\\
	$Z'$ & The negation of the objective function\\
	$n$ & A number in [0, $\mathbb{N}$) representing the rows in the simplex 
	tableau \\
	$m$ & A number in [0, $\mathbb{N}$) representing the columns in the 
	simplex tableau \\
	$x$ & A number in $\mathbb{N}$ representing the size of the list of 
	optimal solutions \\
	\bottomrule
\end{tabular}\\

\newpage

\tableofcontents

\newpage

\pagenumbering{arabic}

\section{Introduction}

The following document details the Module Interface Specifications for the 
\progname{} (\famname{}) tool.

Complementary documents include the System Requirement Specifications and 
Module Guide.  The full documentation and implementation can be found at  
\url{https://github.com/hananezlitni/HZ-CAS741-Project}.

\section{Notation}

The structure of the MIS for modules comes from \citet{HoffmanAndStrooper1995},
with the addition that template modules have been adapted from
\cite{GhezziEtAl2003}.  The mathematical notation comes from Chapter 3 of
\citet{HoffmanAndStrooper1995}.  For instance, the symbol := is used for a
multiple assignment statement and conditional rules follow the form $(c_1
\Rightarrow r_1 | c_2 \Rightarrow r_2 | ... | c_n \Rightarrow r_n )$.

The following table summarizes the primitive data types used by \famname{}. 

\begin{center}
\renewcommand{\arraystretch}{1.2}
\noindent 
\begin{tabular}{l l p{7.5cm}} 
\toprule 
\textbf{Data Type} & \textbf{Notation} & \textbf{Description}\\ 
\midrule
character & char & a single symbol or digit\\
integer & $\mathbb{Z}$ & a number without a fractional component in (-$\infty$, $\infty$) \\
natural number & $\mathbb{N}$ & a number without a fractional component in [1, $\infty$) \\
real & $\mathbb{R}$ & any number in (-$\infty$, $\infty$)\\
\bottomrule
\end{tabular} 
\end{center}

\noindent
The specification of \famname{} uses some derived data types: sequences, 
strings, and tuples. Sequences are lists filled with elements of the same data 
type. Strings are sequences of characters. Tuples contain a list of values, 
potentially of different types. In addition, \famname{} uses functions, which
are defined by the data types of their inputs and outputs. Local functions are
described by giving their type signature followed by their specification.

\section{Module Decomposition}

The following table is taken directly from the Module Guide document for this project.

\begin{table}[h!]
	\centering
	\begin{tabular}{p{0.3\textwidth} p{0.6\textwidth}}
		\toprule
		\textbf{Level 1} & \textbf{Level 2}\\
		\midrule
		
		{Hardware-Hiding Module} & ~ \\
		\midrule
		
		{Behaviour-Hiding Module} & ~ \\
		\midrule
		
		\multirow{3}{0.3\textwidth}{Software Decision Module}
		& Input\\ 
		& Tableau\\
		& Simplex Method Solver\\
		& Output\\
		\bottomrule
	\end{tabular}
	\caption{Module Hierarchy}
	\label{TblMH}
\end{table}

\newpage
~\newpage

\section{MIS of the Input Module} \label{M_IM} 

\subsection{Module}

input

\subsection{Uses}

None

\subsection{Syntax}

\subsubsection{Exported Constants}

None

\subsubsection{Exported Access Programs}

\begin{center}
	\begin{tabular}{p{3cm} p{4cm} p{4cm} p{4cm}}
		\hline
		\textbf{Name} & \textbf{In} & \textbf{Out} & \textbf{Exceptions} \\
		\hline
		readInputs & $\mathbb{R} ^m$, $\mathbb{R} ^{n*m}$, lctEnum$^{n-1}$, 
		gEnum & - & MISSING{\_}INPUT \\
		validateInputs & - & - & INVALID{\_}INPUT \\
		getObjcFunc & - & $\mathbb{R} ^m$ & - \\
		getLCs & - & $\mathbb{R} ^{n*m}$ & - \\
		getLCsType & - & lctEnum$^{n-1}$ & - \\
		getGoal & - & gEnum & - \\
		\hline
	\end{tabular}
\end{center}

\hz{Explanation:}

\hz{1. \textit{lctEnum}: an enumerated type that can be 0 ($\leq$), 1 ($=$) or 
2 ($\geq$) depending on the type of each linear constraint. 
\textit{lctEnum$^{n-1}$} is an array in which the first element is the type of 
the first linear constraint, the second element is the second linear constraint 
type and so on. Its length is the number of rows of the tableau minus the first 
row (because we want to exclude the objective function).}

\hz{2. \textit{gEnum}: an enumerated type representing the LP goal: 0 (min) or 
1 (max).}

\subsection{Semantics}

\subsubsection{State Variables}

None

\subsubsection{Environment Variables}

None

\subsubsection{Assumptions}

None

\subsubsection{Access Routine Semantics}

\noindent 
getObjcFunc():
\begin{itemize}
	\item transition: -
	\item output: \textit{out} := objcFunc:$\mathbb{R} ^m$
	\item exception: -
\end{itemize}

\noindent 
getLCs():
\begin{itemize}
	\item transition: -
	\item output: \textit{out} := LCs:$\mathbb{R} ^{n*m}$
	\item exception: -
\end{itemize}

\noindent 
getLCsType():
\begin{itemize}
	\item transition: -
	\item output: \textit{out} := 
	
	LCsType:lctEnum$^{n-1}$ ; where lctEnum = $\{0,1,2\}$ 
	\item exception: -
\end{itemize}

\noindent 
getGoal():
\begin{itemize}
	\item transition: -
	\item output: \textit{out} := 
	
	goal:gEnum ; where gEnum = $\{0,1\}$
	\item exception: -
\end{itemize}

\noindent 
readInputs(\textit{obFunc}, \textit{lnrConstr}, \textit{lnrConstrT}, 
\textit{goal}):
\begin{itemize}
	\item transition: -
	\item output: -
	\item exception: \textit{exc} := 
	
	At least one input missing \hspace{3.5cm} $\Rightarrow$ MISSING{\_}INPUT
\end{itemize}

\noindent 
validateInputs():
\begin{itemize}
	\item transition: -
	\item output: -
	\item exception: \textit{exc} := 
	
	$\neg$(Last element of LCsType == 2 AND  Last element of LCs == 0) 
	$\Rightarrow$ INVALID{\_}INPUT \\
	
	\hz{Explanation:}
	
	\hz{I'm checking if the non-negativity constraint is present in the LCs 
	matrix. If it isn't, an exception will be generated.}
\end{itemize}

\subsubsection{Local Functions}

None

~\newpage

\section{MIS of the Tableau Module} \label{M_} 

\subsection{Template Module}

tableauADT

\subsection{Uses}

input

\subsection{Syntax}

\subsubsection{Exported Constants}

None

\subsubsection{Exported Access Programs}

\begin{center}
	\begin{tabular}{p{3cm} p{3cm} p{4cm} p{4cm}}
		\hline
		\textbf{Name} & \textbf{In} & \textbf{Out} & \textbf{Exceptions} \\
		\hline
		TableauT & $\mathbb{R}^m,\mathbb{R}^{n*m}$ & TableauT & - \\
		toCanonical & - & - & - \\
		getTableau & - & TableauT & - \\
		getLCsType & - & lctEnum$^{n-1}$ & - \\
		getGoal & - & gEnum & - \\
		updateTableau & operEnum & - & - \\
		setGoal & gEnum & - & - \\
		setLCsType & lctEnum$^{n-1}$ & - & - \\
		setWasMin & boolean & - & - \\
		\hline
	\end{tabular}
\end{center}

\hz{Explanation:}

\hz{\textit{TableauT} is an abstract data type that represents a matrix in 
which the first row is the coefficients of the objective function and the rest 
of the rows are the coefficients of the linear constraints. The constructor 
will receive the coefficient array of the objective function and the 
coefficient matrix of the LCs and will form the simplex tableau.}

\hz{\textit{operEnum} is an enumerated type that represents the type of 
operation that will be performed on the tableau to update its values.}

\subsection{Semantics}

\subsubsection{State Variables}

\begin{itemize}
	\item sTableau:TableauT
	\item LCsType:lctEnum$^{n-1}$ ; where lctEnum = $\{0,1,2\}$
	\item goal:gEnum ; where gEnum = $\{0,1\}$
	\item wasMin:boolean 
\end{itemize}

\hz{Explanation:}

\hz{\textit{wasMin} is a way to tell whether the original LP was a min problem 
or not. If it's a min problem, \textit{goal} state variable will be changed to 
max but information about what it was originally would be lost. Therefore, 
\textit{wasMin} will be set to "True" if the original LP is a min problem and 
will be checked after the optimal solution is calculated in the solver module. 
If \textit{wasMin} is true, $Z$ will be multiplied by -1 because that's the 
optimum of a min problem.}

\subsubsection{Environment Variables}

None

\subsubsection{Assumptions}

None

\subsubsection{Access Routine Semantics}

\noindent 
new TableauT(\textit{objcFunc, lnrConstr}):
\begin{itemize}
	\item transition: -
	\item output: \textit{out} := self
	
	The output consists of a matrix in which the first row is the coefficients 
	of the objective function and the rest of the rows are the coefficients of 
	the linear constraints.
	
	\item exception: -
\end{itemize}

\noindent 
toCanonical():
\begin{itemize}
	\item transition: 
		\begin{enumerate}
			\item $\neg$(self.getGoal() == 1) \hspace{4cm} $\Rightarrow$ 
			self.setWasMin(True) \\
			\hspace*{8.3cm} $\Rightarrow$ self.setGoal(1) \\
			\hspace*{8.3cm} $\Rightarrow$ self.updateTableau(0)
			\item (self.getLCsType() contains 0) \hspace{2.65cm} $\Rightarrow$ 
			self.setLCsType([1,1,1,...,1]) \\
			\hspace*{8.3cm} $\Rightarrow$ self.updateTableau(1)
		\end{enumerate}
	\item output: -
	\item exception: -
\end{itemize}

\noindent 
getTableau():
\begin{itemize}
	\item transition: -
	\item output: \textit{out} := sTableau
	\item exception: -
\end{itemize}

\noindent 
getLCsType():
\begin{itemize}
	\item transition: -
	\item output: \textit{out} := LCsType
	\item exception: -
\end{itemize}

\noindent 
getGoal():
\begin{itemize}
	\item transition: -
	\item output: \textit{out} := goal
	\item exception: -
\end{itemize}

\noindent 
updateTableau(\textit{operation}):
\begin{itemize}
	\item transition: \textit{operation} is of type operEnum, where operEnum = 
	\{0,1\} for negating the objective function row in sTableau and adding 
	slack variables to sTableau, respectively.
	\begin{enumerate}
		\item (operation == 0)  \hspace{3cm} $\Rightarrow$ multiply the first 
		row of sTableau by -1
		\item (operation == 1) \hspace{3cm} $\Rightarrow$ add slack variables 
		coefficients to sTableau
	\end{enumerate}
	\item output: -
	\item exception: -
\end{itemize}

\noindent 
setGoal(\textit{newGoal}):
\begin{itemize}
	\item transition: 
	
	self.goal := newGoal
	\item output: -
	\item exception: -
\end{itemize}

\noindent 
setLCsType(\textit{newLCsType}):
\begin{itemize}
	\item transition: 
	
	self.LCsType := newLCsType
	\item output: -
	\item exception: -
\end{itemize}

\subsubsection{Local Functions}

None.

~\newpage

\section{MIS of the Simplex Method Solver} \label{M_simplexSolver} 

\subsection{Module}

simplexSolver

\subsection{Uses}

tableauADT

\subsection{Syntax}

\subsubsection{Exported Constants}

None

\subsubsection{Exported Access Programs}

\begin{center}
	\begin{tabular}{p{2cm} p{4cm} p{3cm} p{5cm}}
		\hline
		\textbf{Name} & \textbf{In} & \textbf{Out} & \textbf{Exceptions} \\
		\hline
		solveLP & TableauT, boolean & - & NO\_OPTIMAL\_SOLUTION \\
		getZ & - & $R^x$ & - \\
		getK & - & $R^m$ & - \\
		setZ & $R^x$ & - & - \\
		setK & $R^m$ & - & - \\
		\hline
	\end{tabular}
\end{center}

\subsection{Semantics}

\subsubsection{State Variables}

\begin{itemize}
	\item $Z:\mathbb{R}^x$
	\item $K:\mathbb{R}^m$
\end{itemize}

\subsubsection{Environment Variables}

None

\subsubsection{Assumptions}

None

\subsubsection{Access Routine Semantics}

\noindent 
solveLP(\textit{tableau}, \textit{wasMin}):
\begin{itemize}
	\item transition: 
		\begin{enumerate}
			\item findPivot(\textit{tableau})
			\item pivot(\textit{tableau}, \textit{pivotRow}, 
			\textit{pivotColumn})
			\item Repeat 1 and 2 until there are no negative values in the last 
			row (excluding the last column)
			\item setZ(\textit{optimalSolution})
			\item setK(\textit{points})
			%%\item (\textit{wasMin} == True) \hspace{2cm} $\Rightarrow$ 
			%%setZ($Z'$) ; where $Z'$ = -$Z$
		\end{enumerate}

	\item output: -
	\item exception: \textit{exc} := 
	
	($Z$ == NULL) \hspace{3.7cm} $\Rightarrow$ NO\_OPTIMAL\_SOLUTION
\end{itemize}

\subsubsection{Local Functions}

\noindent 
findPivot(\textit{tableau}):

\hspace*{0.3cm} start \\
\hspace*{2cm} for each \textit{entry} in \textit{tableau} except for the last 
column \\
\hspace*{3cm} min(negative \textit{entry}) \\
\hspace*{3cm} if \textit{entry} is found: \\
\hspace*{4cm} \textit{pivotColumn} = j \\
\hspace*{4cm} \textcolor{pyComment}{\#\textit{j is the column where the most 
negative entry was found}} \\
\hspace*{2cm} for each \textit{element} in column \textit{pivotColumn} and 
\textit{constant} in the last column\\
\hspace*{3cm} min(\textit{element} / \textit{constant}) \\
\hspace*{3cm} \textit{pivotRow} = i \\
\hspace*{3cm} \textcolor{pyComment}{\#\textit{i is the row where the most 
minimum ratio was found}} \\
\hspace*{2cm} return \textit{pivotRow, pivotColumn} \\
\hspace*{0.9cm} end \\

\noindent 
pivot(\textit{tableau}, \textit{pivotRow}, \textit{pivotColumn}):

\hspace*{0.3cm} start \\
\hspace*{2cm} for each \textit{row} in \textit{pivotColumn} \\
\hspace*{3cm} perform a row operation to make entry 0 \\
\hspace*{0.9cm} end \\

~\newpage

\section{MIS of the Output Module} \label{M} 

\subsection{Module}

output

\subsection{Uses}

simplexSolver 

\subsection{Syntax}

\subsubsection{Exported Constants}

None

\subsubsection{Exported Access Programs}

\begin{center}
	\begin{tabular}{p{3cm} p{3cm} p{4cm} p{4cm}}
		\hline
		\textbf{Name} & \textbf{In} & \textbf{Out} & \textbf{Exceptions} \\
		\hline
		output & - & - & - \\
		\hline
	\end{tabular}
\end{center}

\subsection{Semantics}

\subsubsection{State Variables}

None

\subsubsection{Environment Variables}

screen: The device screen of the driver program's user

\subsubsection{Assumptions}

None

\subsubsection{Access Routine Semantics}

\noindent 
output():
\begin{itemize}
	\item transition: Display to the environment variable the optimal 
	solution(s) and the points where they occur by calling 
	\textit{simplexSolver.getZ()} and \textit{simplexSolver.getK()}.
	\item output: -
	\item exception: -
\end{itemize}

\subsubsection{Local Functions}

None.

\newpage

\bibliographystyle {plainnat}
\bibliography {../../../refs/References}

\newpage

\section{Appendix} \label{Appendix}

There are no additional information to provide.

\end{document}